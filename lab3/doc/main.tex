
\documentclass[pdf, unicode, 12pt, a4paper,oneside,fleqn]{article}
\usepackage{graphicx}
\graphicspath{{img/}}
\usepackage{log-style}
\begin{document}

\begin{titlepage}
    \begin{center}
        \bfseries

        {\Large Московский авиационный институт\\ (национальный исследовательский университет)}
        
        \vspace{48pt}
        
        {\large Факультет информационных технологий и прикладной математики}
        
        \vspace{36pt}
        
        {\large Кафедра вычислительной математики и~программирования}
        
        \vspace{48pt}
        
        Лабораторная работа \textnumero 3 по курсу \enquote{Операционные системы}

        \vspace{48pt}

        Управление потоками в ОС. Обеспечение синхронизации между потоками
    \end{center}
    
    \vspace{140pt}
    
    \begin{flushright}
    \begin{tabular}{rl}
    Студент: & П.\,Ф. Гришин \\
    Преподаватель: & Е. \,С. Миронов \\
    Группа: & М8О-201Б-21 \\
    Вариант: & 1 \\
    Дата: & \\
    Оценка: & \\
    Подпись: & \\
    \end{tabular}
    \end{flushright}
    
    \vfill
    
    \begin{center}
    \bfseries
    Москва, \the\year
    \end{center}
\end{titlepage}
    
\pagebreak

\section{Постановка задачи}

Составить программу на языке Си, обрабатывающую данные в многопоточном
режиме. При обработки использовать стандартные средства создания потоков
операционной системы (Windows/Unix). Ограничение потоков должно быть задано 
ключом запуска вашей программы.

Так же необходимо уметь продемонстрировать количество потоков, используемое 
вашей программой с помощью стандартных средств операционной системы.

В отчете привести исследование зависимости ускорения и эффективности 
алгоритма от входящих данных и количества потоков. Получившиеся результаты 
необходимо объяснить.

Отсортировать массив целых чисел при помощи битонической сортировки.

\section{Сведения о программе}

Программа написанна на Си в Unix подобной операционной системе на базе ядра Linux.
Для компиляции требуется ключ -lpthread, для запуска программы нелбходимо указать
в качестве аргумента количество потоков, которые максимально могут быть использованы.

Программа сортирует массив с помощью битонной сортировки.
Сначала пользователь должен ввести колличество элементов массива.
Далее должен передать все элементы.

Существует три глобальных переменных, которые являются мьютексом и 2 переменнами колличества потоков,
используемые и максимальные. Их использование между потоками регулируется мьютексом.
Сортируемый массив передается как указатель. И для него не нужен мьютекс, так как 
каждый поток работает на своем участке, не перекрываемом другими.

\section{Общий метод и алгоритм решения}

Параментром запуска программы мы указваем максимальное количество использумых потоков.

Далее мы указываем длину массива, и создаем массив дополив длину до ближайшего числа $2^n$.
Так как это необходимо для работы алгоритма сортировки. Далее мы считываем 
данные, а оставшийся участок массива заполняем максимально возможными элементами,
чтобы после сортировки они оказалить в конце массива, и мы могли также легко их удалить.

Далее вызывается функция сортировки. Сначала мы рекурсивно разбиваем 
массив по полам, задавая целевое направление сначала вниз, а потом вверх, чтобы получить
битонную последовательность. При этом пока у нас есть свободные потоки мы выдиляем под вторую
половину отдельный поток, а когда они закончатся начинаем работать в однопотоке, на выделенном участке.

Когда мы достигаем массива из одного элемента его можно считать отсортированным, как по возрастанию так
и по убыванию. Поэтому два соседних элемента можно считать битонной последовательностью, которую можно собрать в одну
возрастающую или убывающую.

Начинаем объядинять битонные последовательности. Находим расстояние между двумя
элиментами, которые будем сравнивать, оно равно половине участка массива.
Далее мы проходим по половине участка и сравниваем каждый элемент с элементом через заданное расстояние.
При необходимости меняем их местами. Далее разбиваем данный участок на 2 и повторяем слияние, и так пока его длина не станет равна 1.

Таким образом мы получаем битоническую последовательность большего размера, к которой можно опять применить слияние.
Повторяем эту процедуру до оканчания сортироки. При этом когда мы будем делать слияние для 2 участков, созданных
разными потоками, слияние мы опять разбиваем на несколько потоков, пока это возможно.

При всем этом если один поток подготовил последовательность, а второй еще нет, то первый его ждет, из чего
следует вывод, что увеличение производительности будет только при увеличении количества потоков
до следующей степени двойки.

\section{Листинг программы}

{\large\textbf{main.c}}

\begin{lstlisting}[language=C]
#include "library.h"
#include "bitonic.h"
#include "pthread.h"
#include <sys/time.h>

#define INVALID_INPUT 403
#define WRONG_ANSWER 405

#define MAXINT 2147483647
#define UP 1
#define DOWN 0

pthread_mutex_t lock;
size_t max_threads = 1;
size_t use_threads = 1;

void InitArgs(ArgsBitonic *args, int *array, int size, int start, int dir){
    args->array = array;
    args->size  = size;
    args->start = start;
    args->dir   = dir;
}


void Comparator(int *array, int i, int j, int dir){
    if(dir == (array[i] > array[j])){
        int temp = array[i];
        array[i] = array[j];
        array[j] = temp;
    }
}

void BitonicMergeSinglThread(ArgsBitonic *args){
    if(args->size > 1){
        int nextsize = args->size / 2;
        for(int i = args->start; i < nextsize + args->start; ++i){
                Comparator(args->array, i, i + nextsize, args->dir);
        }

        ArgsBitonic args1;
        ArgsBitonic args2;
        InitArgs(&args1, args->array, nextsize, args->start, args->dir);
        InitArgs(&args2, args->array, nextsize, args->start + nextsize, args->dir);
        
        BitonicMergeSinglThread(&args1);
        BitonicMergeSinglThread(&args2);
    }
}

void BitonicSortSinglThread(ArgsBitonic *args){
    if(args->size > 1){
        int nextsize = args->size / 2;

        ArgsBitonic args1;
        ArgsBitonic args2;
        InitArgs(&args1, args->array, nextsize, args->start, DOWN);
        InitArgs(&args2, args->array, nextsize, args->start + nextsize, UP);

        BitonicSortSinglThread(&args1);
        BitonicSortSinglThread(&args2);
        BitonicMergeSinglThread(args);
    }
}

void BitonicMergeMultiThreads(ArgsBitonic *args){
    if(args->size > 1){        
        int nextsize = args->size / 2;
        int isParal = 0;
        pthread_t tid;

        for(int i = args->start; i < nextsize + args->start; ++i){
                Comparator(args->array, i, i + nextsize, args->dir);
        }

        ArgsBitonic args1;
        ArgsBitonic args2;
        InitArgs(&args1, args->array, nextsize, args->start, args->dir);
        InitArgs(&args2, args->array, nextsize, args->start + nextsize, args->dir);

        pthread_mutex_lock(&lock);
        if(use_threads < max_threads){ 
            ++use_threads;
            pthread_mutex_unlock(&lock);
            isParal = 1;
            pthread_create(&tid, NULL,(void*) &BitonicMergeMultiThreads, &args1);
            BitonicMergeMultiThreads(&args2);
        } else {
            pthread_mutex_unlock(&lock);
            BitonicMergeSinglThread(&args1);
            BitonicMergeSinglThread(&args2);
        }

        if(isParal){
            pthread_join(tid, NULL);
            pthread_mutex_lock(&lock);
            --use_threads;
            pthread_mutex_unlock(&lock);
        }
    }
}
 
void BitonicSortMultiThreads(ArgsBitonic *args){
    if(args->size > 1 ){
        int nextsize = args->size / 2;
        int isParal = 0;
        pthread_t tid;

        ArgsBitonic args1;
        ArgsBitonic args2;
        InitArgs(&args1, args->array, nextsize, args->start, DOWN);
        InitArgs(&args2, args->array, nextsize, args->start + nextsize, UP);

        pthread_mutex_lock(&lock);
        if(use_threads < max_threads){
            ++use_threads;
            pthread_mutex_unlock(&lock);
            isParal = 1;
            pthread_create(&tid, NULL,(void*) &BitonicSortMultiThreads, &args1);
            BitonicSortMultiThreads(&args2);
        } else {
            pthread_mutex_unlock(&lock);
            BitonicSortSinglThread(&args1);
            BitonicSortSinglThread(&args2);
        }

        if(isParal){
            pthread_join(tid, NULL);
            pthread_mutex_lock(&lock);
            --use_threads;
            pthread_mutex_unlock(&lock);
        }
        BitonicMergeMultiThreads(args);
    }
}

void bitonicsort(int *array, int size, int threads){
    pthread_mutex_init(&lock, NULL);

    ArgsBitonic args;
    InitArgs(&args,array,size,0,UP);

    if(threads > 1)
        max_threads = threads;

    BitonicSortMultiThreads(&args);
    
    pthread_mutex_destroy(&lock);
}

int SizeStep(int Num){
    int i = 1;
    while(i < Num)
        i *= 2;
    return i;
}


int ThreadSort(int threads, char* nameF){
    FILE* file;
    if(threads < 1){
        printf("ERROR: Incorrect number of threads\n");
        return INVALID_INPUT;
    }

    int input_size;
    file = fopen(nameF, "r");
    if (file == NULL) printf("ERROR: Can't open this file\n");

    fscanf(file, "%d", &input_size);
    //находим ближашее число 2^k >= input_size
    int size_array = SizeStep(input_size);
    int *array = malloc(sizeof(int)*size_array);

    for(int i = 0; i < input_size; ++i)
        fscanf(file,"%d", array+i);

    fclose(file);

    for(int i = input_size; i < size_array; ++i)
        array[i] = MAXINT;


    bitonicsort(array, size_array, threads);

    for(int i = 1; i < input_size; ++i) {
        if(array[i - 1] > array[i]) {
            printf("ERROR: Something goes wrong. Array is not sorted\n");
            return WRONG_ANSWER;
        }
    }
    printf("Array is sorted\n");

    free(array);
    return 0;
}
\end{lstlisting}

\section{Демонстрация работы программы}

\begin{alltt}
gpavel@gpavel-HP-Pavilion-Gaming-Laptop-17-cd1xxx:~/Desktop/OS/tests$ ./lab3_test
Running main() from /home/gpavel/Desktop/tmpo/tests/_deps/googletest-src/googletest/srс
[==========] Running 3 tests from 2 test suites.
[----------] Global test environment set-up.
[----------] 2 tests from ThirdLabTests
[ RUN      ] ThirdLabTests.IncorrectNumberOfThreads
ERROR: Incorrect number of threads
ERROR: Incorrect number of threads
Passed the test for incorrect input of the number of threads
[       OK ] ThirdLabTests.IncorrectNumberOfThreads (0 ms)
[ RUN      ] ThirdLabTests.SingleThread
Array is sorted
[       OK ] ThirdLabTests.SingleThread (1085 ms)
[----------] 2 tests from ThirdLabTests (1085 ms total)
[----------] 1 test from ThirdLabTest
[ RUN      ] ThirdLabTest.MultiThreads
Max thread count is 2
Array is sorted
Runtime is 604 ms

Max thread count is 3
Array is sorted
Runtime is 581 ms

Max thread count is 4
Array is sorted
Runtime is 567 ms

Max thread count is 5
Array is sorted
Runtime is 536 ms

7Max thread count is 6
Array is sorted
Runtime is 555 ms

Max thread count is 7
Array is sorted
Runtime is 551 ms

Max thread count is 8
Array is sorted
Runtime is 421 ms

Max thread count is 9
Array is sorted
Runtime is 362 ms
[       OK ] ThirdLabTest.MultiThreads (4182 ms)
[----------] 1 test from ThirdLabTest (4182 ms total)
[----------] Global test environment tear-down
[==========] 3 tests from 2 test suites ran. (5267 ms total)
[ PASSED ] 3 tests.
gpavel@gpavel-HP-Pavilion-Gaming-Laptop-17-cd1xxx:~/Desktop/OS/tests$
\end{alltt}
\pagebreak

\section{Вывод}

Многие языки программирования позволяют пользователю работать с потоками. 
Создание потоков происходит быстрее, чем создание процессов, за счет того, что
при создании потока не копируется область памяти, а они все работают с одной
областью памяти. Поэтому многопоточность используют для ускарения не зависящих
друг от друга, однотипнях задач, которые будут работать параллельно.

Язык Си предоставляет данный функционал пользователям Unix-подобных операционных
систем с помощью библиотеки pthread.h. Средствами языка Си можно совершать системные
запросы на создание и ожидания завершения потока, а также использовать различные
примитивы синхронизации.

В данной лабораторной работе был реализован и исследован алгоритм битонной сортировки.
Установив при этом, что используя 9 потока можно молучить выигрыш по времени
в 2 раза. При дальнейшем увеличении потоков прирост почти не увеличиватся, а
даже может уменьшаться из-за того, что на управление и переключение потоков 
уходит больше времени, чем они выигрывают.
\end{document}

