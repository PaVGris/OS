
\documentclass[pdf, unicode, 12pt, a4paper,oneside,fleqn]{article}
\usepackage{graphicx}
\graphicspath{{img/}}
\usepackage{log-style}
\begin{document}

\begin{titlepage}
    \begin{center}
        \bfseries

        {\Large Московский авиационный институт\\ (национальный исследовательский университет)}
        
        \vspace{48pt}
        
        {\large Факультет информационных технологий и прикладной математики}
        
        \vspace{36pt}
        
        {\large Кафедра вычислительной математики и~программирования}
        
        \vspace{48pt}
        
        Лабораторная работа \textnumero 3 по курсу \enquote{Операционные системы}

        \vspace{48pt}

        Управление потоками в ОС. Обеспечение синхронизации между потоками
    \end{center}
    
    \vspace{140pt}
    
    \begin{flushright}
    \begin{tabular}{rl}
    Студент: & П.\,Ф. Гришин \\
    Преподаватель: & Е. \,С. Миронов \\
    Группа: & М8О-201Б-21 \\
    Вариант: & 4 \\
    Дата: & \\
    Оценка: & \\
    Подпись: & \\
    \end{tabular}
    \end{flushright}
    
    \vfill
    
    \begin{center}
    \bfseries
    Москва, \the\year
    \end{center}
\end{titlepage}
    
\pagebreak

\section{Постановка задачи}

Составить программу на языке Си, обрабатывающую данные в многопоточном
режиме. При обработки использовать стандартные средства создания потоков
операционной системы (Windows/Unix). Ограничение потоков должно быть задано 
ключом запуска вашей программы.

Так же необходимо уметь продемонстрировать количество потоков, используемое 
вашей программой с помощью стандартных средств операционной системы.

В отчете привести исследование зависимости ускорения и эффективности 
алгоритма от входящих данных и количества потоков. Получившиеся результаты 
необходимо объяснить.

Отсортировать массив целых чисел при помощи битонической сортировки.

\section{Постановка задачи}

Требуется создать динамические библиотеки, которые реализуют определенный функционал. 
Далее использовать данные библиотеки 2-мя способами:

\begin{itemize}
    \item Во время компиляции (на этапе «линковки»/linking)
    \item Во время исполнения программы. Библиотеки загружаются в память с помощью 
            интерфейса ОС для работы с динамическими библиотеками
\end{itemize}

В лабораторной работе необходимо получить следующие части:

\begin{itemize}
    \item Динамические библиотеки, реализующие контракты
    \item Тестовая программа (программа №1), которая используют одну из библиотек, используя 
    знания полученные на этапе компиляции.
    \item Тестовая программа (программа №2), которая загружает библиотеки, используя только их 
    местоположение и контракты.
\end{itemize}

Провести анализ двух типов использования библиотек.

Пользовательский ввод должен быть организован следующим образом:

\begin{itemize}
    \item Команда <<0>>: переключить одну реализацию контракты на другую
    \item Команда <<1 args>>: вызов первой функции контрактов
    \item Команда <<2 args>>: вызов второй функции контрактов
\end{itemize}

Контракты:
\begin{itemize}
    \item Рассчет интеграла функции sin(x) на отрезке [A, B] с шагом e
    \item Рассчет значения числа Пи при заданной длине ряда (K)
\end{itemize}

\section{Сведения о программе}

Программа написанна на Си в Unix подобной операционной системе на базе ядра Linux.

Контракты описаны в файле functions.h, а реализация functions\_1.с и functions\_2.с.

\begin{enumerate}
    \item Создание объектных файлов
    \item Компиляция библиотек с ключем -shared. Получаем динамические библиотеки с расширением .so 
    \item Линковка библиотеки к необходимой программе
\end{enumerate}

Для динамической загрузки библиотек используется библиотека dlfcn.h

\section{Общий метод и алгоритм решения}

Программа принимает в себя команды: 
\begin{itemize}
    \item В случае команды 1, мы считываем координаты левого и правого отрезка, интеграл функции которого хотим найти, а также шаг e, и находим интеграл либо методом прямоугольников, либо методом трапеций.
    \item В случае команды 2, мы считываем число - порядок ряда. Находим его либо с помощью ряда Лейбница, либо с помощью формулы Валлиса.
    \item В случае команды 0, мы закрываем старую библиотеку, открываем вторую и заменяем указатели на функции.
\end{itemize}

Для завершения программы нужно ввести комбинацию завершения ввода \--- CTRL+D.

\section{Листинг программы}

{\large\textbf{func.h}}

\begin{lstlisting}[language=C]
#ifndef FUNC_H
#define FUNC_H

#include <stdlib.h>
#include <stdio.h>
#include <math.h>

double Integrate(double a, double b, double epsilon);
double Pi(int K);

#endif
\end{lstlisting}

{\large\textbf{lib1.c}}

\begin{lstlisting}[language=C]
double Integrate(double a, double b, double epsilon)
{
    int steps = fabs(b - a) / epsilon;
    double point = a;
    double result = 0;

    for (int i = 0; i < steps; ++i)
    {
        result += sin(point) * epsilon;
        point += epsilon;
    }

    return result;
}

double Pi(int K) {
    double pi = 0;
    for(int i = 0; i <= K; i++) {
        pi += 2.0/((4*i + 1)*(4*i + 3));
    }
    return pi * 4.0;
}
\end{lstlisting}

{\large\textbf{lib2.c}}

\begin{lstlisting}[language=C]
double Integrate(double a, double b, double epsilon)
{
    int steps = fabs(b - a) / epsilon;
    double point = a;
    double result = 0;

    for (int i = 0; i < steps; ++i)
    {
        result += sin(point + epsilon / 2) * epsilon;
        point += epsilon;
    }

    return result;
}

double Pi(int K) {
    double pi = 1;
    for (int i = 1; i <= K; i++) {
        pi *= (double)4*i*i/(4*i*i-1);
    }
    return (double)pi * 2;
}
\end{lstlisting}

{\large\textbf{main\_dynamic.c}}

\begin{lstlisting}[language=C]
#include <stdlib.h>
#include <stdio.h>
#include <dlfcn.h>
#include <math.h>
#include <stdbool.h>

const char LIB1[] = "./libd1_dynamic.so";
const char LIB2[] = "./libd2_dynamic.so";

int main(int argc, char* argv[]) {
    void *library;
    bool flag = false;
    int x, K;
    double a, b, e;

    library = dlopen(LIB2, RTLD_LAZY);
    if (!library) {
        printf("Error dlopen(): %s\n", dlerror());
        return 1;
    }

    double(*Integrate)(double a, double b, double e);
    double(*Pi)(int K);
    *(void**)(&Integrate) = dlsym(library, "Integrate");
    *(void**)(&Pi) = dlsym(library, "Pi");

    for (;;) {
        scanf("%d", &x);
        if (x == 0) {
            dlclose(library);
            if (flag) {
                library = dlopen(LIB2, RTLD_LAZY);
                flag = false;
            } else {
                library = dlopen(LIB1, RTLD_LAZY);
                flag = true;
            }
            if (!library) {
                printf("Error dlopen(): %s\n", dlerror());
                return 1;
            }
            *(void**)(&Integrate) = dlsym(library, "Integrate");
            *(void**)(&Pi) = dlsym(library, "Pi");
        } else if (x == 1) {
            scanf("%lf %lf %lf", &a, &b, &e);
            printf("Result: ");
            double res = Integrate(a, b, e);
            printf("%lf\n", res);
        } else if (x == 2) {
            scanf("%d", &K);
            printf("Result: ");
            double res = Pi(K);
            printf("%lf\n", res);
        } else {
            dlclose(library);
            return 0;
        }
    }
    return 0;
}
\end{lstlisting}

{\large\textbf{main\_static.c}}

\begin{lstlisting}[language=C]
#include "func.h"
#include <stdlib.h>
#include <stdio.h>

int main(int argc, char* argv[]) {
    int x;
    double a, b, e;
    int K;

    for (;;) {
        scanf("%d", &x);
        if (x == 1) {
            scanf("%lf %lf %lf", &a, &b, &e);
            double res = Integrate(a, b, e);
            printf("%lf\n", res);
            x = 0;
        } else if (x == 2) {
            scanf("%d", &K);
            double res = Pi(K);
            printf("%lf\n", res);
            x = 0;
        } else {
            break;
        }
    }
    return 0;
}
\end{lstlisting}

{\large\textbf{CMakeListst.txt}}

\begin{lstlisting}[language=C]
add_library(d1_static STATIC src/lib1.c include/func.h)
add_library(d2_static STATIC src/lib2.c include/func.h)
add_library(d1_dynamic SHARED src/lib1.c include/func.h)
add_library(d2_dynamic SHARED src/lib2.c include/func.h)

add_executable(main_static_1 main_static.c)
add_executable(main_static_2 main_static.c)
add_executable(main_dynamic main_dynamic.c)

target_include_directories(main_static_1 PRIVATE include)
target_include_directories(main_static_2 PRIVATE include)

target_link_libraries(main_dynamic ${CMAKE_DL_LIBS})
target_link_libraries(main_static_1 PRIVATE d1_static)
target_link_libraries(main_static_2 PRIVATE d2_static)
target_include_directories(main_dynamic PRIVATE include)

find_library(MATH_LIBRARY m)

    target_link_libraries(d1_dynamic PUBLIC ${MATH_LIBRARY})
    target_link_libraries(d2_dynamic PUBLIC ${MATH_LIBRARY})
    target_link_libraries(main_static_1 PUBLIC ${MATH_LIBRARY})
    target_link_libraries(main_static_2 PUBLIC ${MATH_LIBRARY})
    target_link_libraries(main_dynamic PUBLIC ${MATH_LIBRARY})
\end{lstlisting}

\pagebreak

\section{Демонстрация работы программы}

\begin{alltt}
pavel@DESKTOP-K5KMLPV:~/Project/mai/2_course/OS/LB5$ gpavel@gpavel-HP-Pavilion-Gaming-Laptop-17-cd1xxx:~/Desktop/OS/lab5$ ./main_static_1
1
0
3.14
0.1
1.995390
2
10
3.096162
^C
gpavel@gpavel-HP-Pavilion-Gaming-Laptop-17-cd1xxx:~/Desktop/OS/lab5$ ./main_static_2
1
0
3.14
0.1
1.999968
2
10
3.067704
^C
gpavel@gpavel-HP-Pavilion-Gaming-Laptop-17-cd1xxx:~/Desktop/OS/lab5$ ./main_dynamic
1
0
3.14
0.1
Result: 1.999968
2
10
Result: 3.067704
0
1
0
3.14
0.1
Result: 1.995390
2
10
Result: 3.096162
^C
gpavel@gpavel-HP-Pavilion-Gaming-Laptop-17-cd1xxx:~/Desktop/OS/lab5$
\end{alltt}

\pagebreak

\section{Вывод}

Лабораторная работа была направлена на изучение динамических библиотек в Unix подобных операционных
системах. Для изучения создания и работы с ними мною было написанно 2 программ: одна
подлючала динамичкие библиотеки на этапе компиляции, а вторая во время исполнения.

Динамические библиотеки содержат функционал отдельно от программы и прередают его
непосредственно во время исполнения. Из плюсов такого подхода можно выделить, что 
во-первых, в таком случае размер результирующей программы меньше,во-вторых, одну и ту же библиотеку 
можно использовать в нескольких программах не встраивая в код, чем можно также добиться снижения
общего занимаемого пространства на диске, и в-третьих, что после исправления ошибок
в библиотеке не нужно перекомпилировать все программы, достаточно перекомпилировать саму библиотеку.

Однако у динамических библиотек есть и недостатки. Первый заключается в том, что
вызов функции из динамической библиотеки происходит медленнее. Второй, что
мы не можем подправить функционал библиотеки под конкретную программу не зацепив
при этом других программ, работающих с этой библиотекой. И в-третьих, уже скомпилированная
программа не будет работать на аналогичной системе без установленной динамической библиотеки.

Тем не менее плюсы динамичеких библиотек исчерпывают их минусы в большенстве задач, 
обратных случаях лучше обратиться к статическим библиотекам. В наше время с высокими мощностями 
вычислительных систем становится более важным сэкономить объем памяти, используемый программой,
чем время обращения к функции. Поэтому динамические библиотеки используются в большенстве 
современных программ.
\end{document}
\end{document}

