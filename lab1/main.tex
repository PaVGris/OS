
\documentclass[pdf, unicode, 12pt, a4paper,oneside,fleqn]{article}
\usepackage{graphicx}
\graphicspath{{img/}}
\usepackage{log-style}
\begin{document}

\begin{titlepage}
    \begin{center}
        \bfseries

        {\Large Московский авиационный институт\\ (национальный исследовательский университет)}
        
        \vspace{48pt}
        
        {\large Факультет информационных технологий и прикладной математики}
        
        \vspace{36pt}
        
        {\large Кафедра вычислительной математики и~программирования}
        
        \vspace{48pt}
        
        Лабораторная работа \textnumero 1 по курсу \enquote{Операционные системы}

        \vspace{48pt}

        Приобретение практических навыков диагностики работы программного обеспечения.
    \end{center}
    
    \vspace{150pt}
    
    \begin{flushright}
    \begin{tabular}{rl}
    Студент: & П.\,Ф. Гришин \\
    Преподаватель: & Е.\,С. Миронов \\
    Группа: & М8О-201Б-21 \\
    Дата: & \\
    Оценка: & \\
    Подпись: & \\
    \end{tabular}
    \end{flushright}
    
    \vfill
    
    \begin{center}
    \bfseries
    Москва, \the\year
    \end{center}
\end{titlepage}
    
\pagebreak

\section{Постановка задачи}

При выполнении последующих лабораторных работ необходимо продемонстрировать 
ключевые системные вызовы, которые в них используются.

Используемые утилиты:strace, ltrace.

\section{Описание утилит}

Strace \--- утилита Linux для отслеживания системных вызовов, которые представляют собой
механизм трансляции, обеспечивающий интерфейс между процессом и ядром. Работы strace возможна
благодаря функции ядра ptrace. С помощью данной утилиты можно понять, что процесс пытается сделать в данное время.

Используемые ключи:

\begin{itemize}
    \item -o file \--- перенаправить вывод протокола в файл
    \item -f \--- отслежичать системные вызовы у дочерних процессов
    \item -e trace=filters \--- указываются группы системных вызовов по которым будет фильтроваться вывод.
\end{itemize}

ltrace \--- утилита Linux, отслеживающая системные вызовы, связанные с динамическими библиотеками.

\section{Примеры использования утилит}

{\large\textbf{Лабораторная работа 2}}

\begin{alltt}
gpavel@gpavel-HP-Pavilion-Gaming-Laptop-17-cd1xxx:~/Desktop/OS/lab2$ strace -f -e  trace="read,write,dup2,pipe" -o log.txt ./Lab2 
test.txt
[60645] PARENT. Enter the name of file to write: Output.txt
[60645] PARENT. Enter string: CCCCCC.
[60645] PARENT. Enter string: WrongInput
[60645] PARENT. Error: "WrongInput" is not valid.
[60645] PARENT. Enter string: Test
[60645] PARENT. Error: "Test" is not valid.
[60645] PARENT. Enter string: Failed
[60645] PARENT. Error: "Failed" is not valid.
[60645] PARENT. Enter string: LOL;

[60645] PARENT. Enter string: gpavel@gpavel-HP-Pavilion-Gaming-Laptop-17-cd1xxx:~/Desktop/OS/lab2$ cat log.txt 
60645 read(3, "\177ELF\2\1\1\3\0\0\0\0\0\0\0\0\3\0>\0\1\0\0\0P\237\2\0\0\0\0\0"..., 832) = 832
60645 read(0, "test.txt\n", 1024)       = 9
60645 read(3, "Output.txt\nCCCCCC.\nWrongInput\nTe"..., 4096) = 47
60645 write(1, "[60645] PARENT. Enter the name o"..., 60) = 60
60645 write(5, "Output.txt\0002\37\177\0\0\0\0\0\0\0\0\0\0\0\0\0\0\0\0\0\0"..., 256) = 256
60647 read(4,  <unfinished ...>
60645 read(6,  <unfinished ...>
60647 <... read resumed>"Output.txt\0002\37\177\0\0\0\0\0\0\0\0\0\0\0\0\0\0\0\0\0\0"..., 64) = 64
60647 read(8, "\177ELF\2\1\1\3\0\0\0\0\0\0\0\0\3\0>\0\1\0\0\0P\237\2\0\0\0\0\0"..., 832) = 832
60647 write(7, "\0", 1)                 = 1
60645 <... read resumed>"\0", 1)        = 1
60647 read(4,  <unfinished ...>
60645 write(1, "[60645] PARENT. Enter string: CC"..., 38 <unfinished ...>
60647 <... read resumed>"\30\301#2\37\177\0\0\0\310#2\37\177\0\0\1\0\0\0\376\177\0\0\351\376\"2\37\177\0\0"..., 256) = 192
60645 <... write resumed>)              = 38
60647 read(4,  <unfinished ...>
60645 write(5, "CCCCCC.\0\0\0\0\0\0\0\0\0\370\303D\241\376\177\0\0i#\341\323\315U\0\0"..., 256) = 256
60647 <... read resumed>"CCCCCC.\0\0\0\0\0\0\0\0\0\370\303D\241\376\177\0\0i#\341\323\315U\0\0"..., 256) = 256
60645 read(6,  <unfinished ...>
60647 write(8, "CCCCCC.", 7)            = 7
60647 write(8, "\n", 1)                 = 1
60647 write(7, "\0", 1 <unfinished ...>
60645 <... read resumed>"\0", 1)        = 1
60647 <... write resumed>)              = 1
60645 write(1, "[60645] PARENT. Enter string: Wr"..., 41 <unfinished ...>
60647 read(4,  <unfinished ...>
60645 <... write resumed>)              = 41
60645 write(5, "WrongInput\0\0\0\0\0\0\370\303D\241\376\177\0\0i#\341\323\315U\0\0"..., 256) = 256
60647 <... read resumed>"WrongInput\0\0\0\0\0\0\370\303D\241\376\177\0\0i#\341\323\315U\0\0"..., 256) = 256
60645 read(6,  <unfinished ...>
60647 write(7, "\1", 1 <unfinished ...>
60645 <... read resumed>"\1", 1)        = 1
60647 <... write resumed>)              = 1
60645 write(1, "[60645] PARENT. Error: \"WrongInp"..., 50 <unfinished ...>
60647 read(4,  <unfinished ...>
60645 <... write resumed>)              = 50
60645 write(1, "[60645] PARENT. Enter string: Te"..., 35) = 35
60645 write(5, "Test\0Input\0\0\0\0\0\0\370\303D\241\376\177\0\0i#\341\323\315U\0\0"..., 256) = 256
60647 <... read resumed>"Test\0Input\0\0\0\0\0\0\370\303D\241\376\177\0\0i#\341\323\315U\0\0"..., 256) = 256
60645 read(6,  <unfinished ...>
60647 write(7, "\1", 1 <unfinished ...>
60645 <... read resumed>"\1", 1)        = 1
60647 <... write resumed>)              = 1
60645 write(1, "[60645] PARENT. Error: \"Test\" is"..., 44 <unfinished ...>
60647 read(4,  <unfinished ...>
60645 <... write resumed>)              = 44
60645 write(1, "[60645] PARENT. Enter string: Fa"..., 37) = 37
60645 write(5, "Failed\0put\0\0\0\0\0\0\370\303D\241\376\177\0\0i#\341\323\315U\0\0"..., 256) = 256
60647 <... read resumed>"Failed\0put\0\0\0\0\0\0\370\303D\241\376\177\0\0i#\341\323\315U\0\0"..., 256) = 256
60645 read(6,  <unfinished ...>
60647 write(7, "\1", 1 <unfinished ...>
60645 <... read resumed>"\1", 1)        = 1
60647 <... write resumed>)              = 1
60645 write(1, "[60645] PARENT. Error: \"Failed\" "..., 46 <unfinished ...>
60647 read(4,  <unfinished ...>
60645 <... write resumed>)              = 46
60645 write(1, "[60645] PARENT. Enter string: LO"..., 35) = 35
60645 write(5, "LOL;\0d\0put\0\0\0\0\0\0\370\303D\241\376\177\0\0i#\341\323\315U\0\0"..., 256) = 256
60647 <... read resumed>"LOL;\\0d\\0put\\0\\0\\0\\0\\0\\0\\370\\303D\\241\\376\\177\\0\\0i#\\341\\323\\315U\\0\\0"..., 256) = 256
60645 read(6,  <unfinished ...>
60647 write(8, "LOL;", 4)               = 4
60647 write(8, "\n", 1)                 = 1
60647 write(7, "\0", 1 <unfinished ...>
60645 <... read resumed>"\0", 1)        = 1
60647 <... write resumed>)              = 1
60645 read(3,  <unfinished ...>
60647 read(4,  <unfinished ...>
60645 <... read resumed>"", 4096)       = 0
60645 write(5, "_quit\0\n\0\0\1\33\3;P\0\0\0\t\0\0\0t\357\377\377\204\0\0\0\244\360\377"..., 256) = 256
60647 <... read resumed>"_quit\0\n\0\0\1\33\3;P\0\0\0\t\0\0\0t\357\377\377\204\0\0\0\244\360\377"..., 256) = 256
60645 write(1, "\n\0", 2)               = 2
60645 write(1, "[60645] PARENT. Enter string: ", 30) = 30
60647 +++ exited with 0 +++
60645 +++ exited with 0 +++
gpavel@gpavel-HP-Pavilion-Gaming-Laptop-17-cd1xxx:~/Desktop/OS/lab2$ 

\end{alltt}

\pagebreak

\section{Вывод}

Strace и ltrace \--- удобные инструменты для системных вызовов, они полезны приотладке и тестировании программ.
Несмотря на то, что протоколы по умолчанию кажутся слишком объемными, информацию в 
них можно отфильтровать с помощью ключей.

Утилита strace позволяет отслеживать большинство системных вызовов, тем не менее если нужно
более точно пронаблюдать за узкими аспектами существуют и другие утилиты, такие как:
ltrace и ftrace. Такие утилиты пригодились мне для изучения работы программ и поиска ошибок в моих лабораторных работах.

\end{document}

